%%%%%%%%%%%%%%%%%%%%%%%%%%%%%%%%%%%%%%%%%%%%%%%%%%%%%%%%%%%%%%%%%%%%%%%%
%%%%%%%%%%%%%%%%%%%%%% Simple LaTeX CV Template %%%%%%%%%%%%%%%%%%%%%%%%
%%%%%%%%%%%%%%%%%%%%%%%%%%%%%%%%%%%%%%%%%%%%%%%%%%%%%%%%%%%%%%%%%%%%%%%%

%%%%%%%%%%%%%%%%%%%%%%%%%%%%%%%%%%%%%%%%%%%%%%%%%%%%%%%%%%%%%%%%%%%%%%%%
%% NOTE: If you find that it says                                     %%
%%                                                                    %%
%%                           1 of ??                                  %%
%%                                                                    %%
%% at the bottom of your first page, this means that the AUX file     %%
%% was not available when you ran LaTeX on this source. Simply RERUN  %%
%% LaTeX to get the ``??'' replaced with the number of the last page  %%
%% of the document. The AUX file will be generated on the first run   %%
%% of LaTeX and used on the second run to fill in all of the          %%
%% references.                                                        %%
%%%%%%%%%%%%%%%%%%%%%%%%%%%%%%%%%%%%%%%%%%%%%%%%%%%%%%%%%%%%%%%%%%%%%%%%

%%%%%%%%%%%%%%%%%%%%%%%%%%%% Document Setup %%%%%%%%%%%%%%%%%%%%%%%%%%%%

% Don't like 10pt? Try 11pt or 12pt
\documentclass[10pt]{article}

% The automated optical recognition software used to digitize resume
% information works best with fonts that do not have serifs. This
% command uses a sans serif font throughout. Uncomment both lines (or at
% least the second) to restore a Roman font (i.e., a font with serifs).
%\usepackage{times}
%\renewcommand{\familydefault}{\sfdefault}

% This is a helpful package that puts math inside length specifications
\usepackage{calc}
\usepackage{comment}

% Simpler bibsection for CV sections
% (thanks to natbib for inspiration)
\makeatletter
\newlength{\bibhang}
\setlength{\bibhang}{1em} %1em}
\newlength{\bibsep}
 {\@listi \global\bibsep\itemsep \global\advance\bibsep by\parsep}
\newenvironment{bibsection}%
        {\begin{enumerate}{}{%
%        {\begin{list}{}{%
       \setlength{\leftmargin}{\bibhang}%
       \setlength{\itemindent}{-\leftmargin}%
       \setlength{\itemsep}{\bibsep}%
       \setlength{\parsep}{\z@}%
        \setlength{\partopsep}{0pt}%
        \setlength{\topsep}{0pt}}}
        {\end{enumerate}\vspace{-.6\baselineskip}}
%        {\end{list}\vspace{-.6\baselineskip}}
\makeatother

% Layout: Puts the section titles on left side of page
\reversemarginpar

%
%         PAPER SIZE, PAGE NUMBER, AND DOCUMENT LAYOUT NOTES:
%
% The next \usepackage line changes the layout for CV style section
% headings as marginal notes. It also sets up the paper size as either
% letter or A4. By default, letter was used. If A4 paper is desired,
% comment out the letterpaper lines and uncomment the a4paper lines.
%
% As you can see, the margin widths and section title widths can be
% easily adjusted.
%
% ALSO: Notice that the includefoot option can be commented OUT in order
% to put the PAGE NUMBER *IN* the bottom margin. This will make the
% effective text area larger.
%
% IF YOU WISH TO REMOVE THE ``of LASTPAGE'' next to each page number,
% see the note about the +LP and -LP lines below. Comment out the +LP
% and uncomment the -LP.
%
% IF YOU WISH TO REMOVE PAGE NUMBERS, be sure that the includefoot line
% is uncommented and ALSO uncomment the \pagestyle{empty} a few lines
% below.
%

%% Use these lines for letter-sized paper
\usepackage[paper=letterpaper,
            %includefoot, % Uncomment to put page number above margin
            marginparwidth=1.2in,     % Length of section titles
            marginparsep=.05in,       % Space between titles and text
            margin=1in,               % 1 inch margins
            includemp]{geometry}

%% Use these lines for A4-sized paper
%\usepackage[paper=a4paper,
%            %includefoot, % Uncomment to put page number above margin
%            marginparwidth=30.5mm,    % Length of section titles
%            marginparsep=1.5mm,       % Space between titles and text
%            margin=25mm,              % 25mm margins
%            includemp]{geometry}

%% More layout: Get rid of indenting throughout entire document
\setlength{\parindent}{0in}

\usepackage[shortlabels]{enumitem}

%% Reference the last page in the page number
%
% NOTE: comment the +LP line and uncomment the -LP line to have page
%       numbers without the ``of ##'' last page reference)
%
% NOTE: uncomment the \pagestyle{empty} line to get rid of all page
%       numbers (make sure includefoot is commented out above)
%
\usepackage{fancyhdr,lastpage}
\pagestyle{fancy}
%\pagestyle{empty}      % Uncomment this to get rid of page numbers
\fancyhf{}\renewcommand{\headrulewidth}{0pt}
\fancyfootoffset{\marginparsep+\marginparwidth}
\newlength{\footpageshift}
\setlength{\footpageshift}
          {0.5\textwidth+0.5\marginparsep+0.5\marginparwidth-2in}
\lfoot{\hspace{\footpageshift}%
       \parbox{4in}{\, \hfill %
                    \arabic{page} of \protect\pageref*{LastPage} % +LP
%                    \arabic{page}                               % -LP
                    \hfill \,}}

% Finally, give us PDF bookmarks
\usepackage{color,hyperref}
\definecolor{darkblue}{rgb}{0.0,0.0,0.3}
\hypersetup{colorlinks,breaklinks,
            linkcolor=darkblue,urlcolor=darkblue,
            anchorcolor=darkblue,citecolor=darkblue}

%%%%%%%%%%%%%%%%%%%%%%%% End Document Setup %%%%%%%%%%%%%%%%%%%%%%%%%%%%


%%%%%%%%%%%%%%%%%%%%%%%%%%% Helper Commands %%%%%%%%%%%%%%%%%%%%%%%%%%%%

% The title (name) with a horizontal rule under it
% (optional argument typesets an object right-justified across from name
%  as well)
%
% Usage: \makeheading{name}
%        OR
%        \makeheading[right_object]{name}
%
% Place at top of document. It should be the first thing.
% If ``right_object'' is provided in the square-braced optional
% argument, it will be right justified on the same line as ``name'' at
% the top of the CV. For example:
%
%       \makeheading[\emph{Curriculum vitae}]{Your Name}
%
% will put an emphasized ``Curriculum vitae'' at the top of the document
% as a title. Likewise, a picture could be included:
%
%   \makeheading[\includegraphics[height=1.5in]{my_picutre}]{Your Name}
%
% the picture will be flush right across from the name.
\newcommand{\makeheading}[2][]%
        {\hspace*{-\marginparsep minus \marginparwidth}%
         \begin{minipage}[t]{\textwidth+\marginparwidth+\marginparsep}%
             {\large \bfseries #2 \hfill #1}\\[-0.15\baselineskip]%
                 \rule{\columnwidth}{1pt}%
         \end{minipage}}

% The section headings
%
% Usage: \section{section name}
\renewcommand{\section}[1]{\pagebreak[3]%
    \hyphenpenalty=1000%
    \vspace{1.3\baselineskip}%
    \phantomsection\addcontentsline{toc}{section}{#1}%
    \noindent\llap{\scshape\smash{\parbox[t]{\marginparwidth}{\raggedright #1}}}%
    \vspace{-\baselineskip}\par}

% An itemize-style list with lots of space between items
\newenvironment{outerlist}[1][\enskip\textbullet]%
        {\begin{itemize}[#1,leftmargin=*]}{\end{itemize}%
         \vspace{-.6\baselineskip}}

% An environment IDENTICAL to outerlist that has better pre-list spacing
% when used as the first thing in a \section
\newenvironment{lonelist}[1][\enskip\textbullet]%
        {\begin{list}{#1}{%
        \setlength{\partopsep}{0pt}%
        \setlength{\topsep}{0pt}}}
        {\end{list}\vspace{-.6\baselineskip}}

% An itemize-style list with little space between items
\newenvironment{innerlist}[1][\enskip\textbullet]%
        {\begin{itemize}[#1,leftmargin=*,parsep=0pt,itemsep=0pt,topsep=0pt,partopsep=0pt]}
        {\end{itemize}}

% An environment IDENTICAL to innerlist that has better pre-list spacing
% when used as the first thing in a \section
\newenvironment{loneinnerlist}[1][\enskip\textbullet]%
        {\begin{itemize}[#1,leftmargin=*,parsep=0pt,itemsep=0pt,topsep=0pt,partopsep=0pt]}
        {\end{itemize}\vspace{-.6\baselineskip}}

% To add some paragraph space between lines.
% This also tells LaTeX to preferably break a page on one of these gaps
% if there is a needed pagebreak nearby.
\newcommand{\blankline}{\quad\pagebreak[3]}
\newcommand{\halfblankline}{\quad\vspace{-0.5\baselineskip}\pagebreak[3]}

% Uses hyperref to link DOI
\newcommand\doilink[1]{\href{http://dx.doi.org/#1}{#1}}
\newcommand\doi[1]{doi:\doilink{#1}}

% For \url{SOME_URL}, links SOME_URL to the url SOME_URL
\providecommand*\url[1]{\href{#1}{#1}}
% Same as above, but pretty-prints SOME_URL in teletype fixed-width font
\renewcommand*\url[1]{\href{#1}{\texttt{#1}}}

% For \email{ADDRESS}, links ADDRESS to the url mailto:ADDRESS
\providecommand*\email[1]{\href{mailto:#1}{#1}}
% Same as above, but pretty-prints ADDRESS in teletype fixed-width font
%\renewcommand*\email[1]{\href{mailto:#1}{\texttt{#1}}}

%\providecommand\BibTeX{{\rm B\kern-.05em{\sc i\kern-.025em b}\kern-.08em
%    T\kern-.1667em\lower.7ex\hbox{E}\kern-.125emX}}
%\providecommand\BibTeX{{\rm B\kern-.05em{\sc i\kern-.025em b}\kern-.08em
%    \TeX}}
\providecommand\BibTeX{{B\kern-.05em{\sc i\kern-.025em b}\kern-.08em
    \TeX}}
\providecommand\Matlab{\textsc{Matlab}}

%%%%%%%%%%%%%%%%%%%%%%%% End Helper Commands %%%%%%%%%%%%%%%%%%%%%%%%%%%

%%%%%%%%%%%%%%%%%%%%%%%%% Begin CV Document %%%%%%%%%%%%%%%%%%%%%%%%%%%%

\begin{document}
\makeheading{Henggang Cui}

\section{Contact Information}

% NOTE: Mind where the & separators and \\ breaks are in the following
%       table.
%
% ALSO: \rcollength is the width of the right column of the table
%       (adjust it to your liking; default is 1.85in).
%
\newlength{\rcollength}\setlength{\rcollength}{1.4in}%
%
\begin{tabular}[t]{@{}p{\textwidth-\rcollength}p{\rcollength}}
%\href{http://www.cse.osu.edu/}%
%     {Department of Computer Science and Engineering} & \\
%\href{http://www.osu.edu/}{The Ohio State University}
2220A, Collaborative Innovation Center,       & (651)403-3366 \\
5000 Forbes Ave, Pittsburgh, PA, 15213       & \email{hengganc@andrew.cmu.edu}\\
\end{tabular}

%\section{Objective}

%Insert text here if you want to
%\begin{innerlist}
%\item More information and auxiliary documents can be found at\\\url{http://www.tedpavlic.com/facjobsearch/}
%\end{innerlist}

\section{Research Interests}

{\bf System Support for Large-Scale Machine Learning}

\section{Education}

\textbf{Carnegie Mellon University},
Pittsburgh, PA.
\begin{outerlist}

\item[] Ph.D.,
             {Electrical and Computer Engineering}, \emph{Expected:} May, 2017
        \begin{innerlist}
        %\item Thesis Topic: \emph{Spatiotemporal Gradient Modeling with Applications}
        \item Advisor: {Greg Ganger}
        \item Thesis Title: {Exploiting Application Characteristics for Efficient System Support of Data-Parallel Machine Learning}
        \end{innerlist}

\end{outerlist}
\vspace{.1in}
%\href{http://www.mnsu.edu}
{\textbf{Tsinghua University}},
Beijing, China
\begin{outerlist}
\item[] B.S.,
        %\href{http://www.cset.mnsu.edu/mathstat/}
             {Electronic Information Science and Technology}, July 2012
        % \begin{innerlist}
        % \item {Outstanding Graduate Award}
        % \item {Outstanding Diploma Thesis Award}
        % \end{innerlist}

\end{outerlist}

\section{Research Experience}

\textbf{Research Assistant} \hfill {Aug 2012 to present}\\
Parallel Data Lab, Carnegie Mellon University\\
Advisor: Greg Ganger
\begin{innerlist}
% \item[] {\bf LazyTable: a parameter server system for data-parallel machine learning.}
    % LazyTable manages the global model parameter data for data-parallel ML applications, where a collection of ML workers concurrently iterate on partitioned training data and make adjustments to the shared parameter data. We have implemented several representative ML applications on top of LazyTable, including image classification (with convolutional neural networks), movie recommendation (with matrix factorization), topic modeling (with LDA), and PageRank.
% \vspace{.1in}
\item[] {\bf Stale Synchronous Parallel (SSP): Trading Data Freshness for Speed.}
    SSP is a flexible consistency model for synchronizing the progress of parallel ML workers. It is a middle ground between the traditional BSP approach (where the workers wait at barriers) and the Asynchrony approach (where the workers never wait). SSP allows each worker to be a bounded number of iterations ahead of the slowest one, thus making data staleness a tunable parameter and allowing us to explicitly trade data freshness for speed. Our experiments show that SSP helps reduce the convergence time of many ML benchmarks, such as matrix factorization, LDA, and PageRank. \\
    \emph{Work published at NIPS'13 and ATC'14.}
\vspace{.1in}
\item[] {\bf IterStore: Efficient Parameter Server for Iterative ML.}
    Many ML algorithms, including SGD, PageRank, and Gibbs Sampling, have the \emph{iterativeness} property that the same (or nearly the same) sequence of parameter data accesses is applied to the parameter server every iteration. This repeating access sequence can be exploited to improve the parameter server throughput. We designed an efficient method to collect the repeating access sequence from the application, as well as five parameter server optimizations using the collected information, including prefetching, contiguous marshalling-free data storage, locality and numa-aware data placement, and specialized caching policy. Our experiments show that these optimizations greatly reduce the total run time of our application benchmarks by up to 50$\times$. \\
    \emph{Work published at SoCC'14.}
\vspace{.1in}
\item[] {\bf GeePS: Specialized Parameter Server for Deep Learning on GPUs.}
    Deep learning tasks are often performed on GPUs, but scaling GPU applications to multiple GPU machines is challenging, because of the limited GPU memory capacity and expensive data movement overheads between GPU and CPU memory. GeePS is a parameter server that is specialized for GPU applications (especially deep learning). It allows the application to access parameter data directly through GPU memory, hiding data movement latencies from the application. GeePS also provides GPU memory management support that allows the training of deep neural networks that do not fit in GPU memory, by efficiently swapping data to/from CPU memory in the background. GeePS, as a parameter server system, can be used to scale many single-machine GPU-based ML systems, such as Caffe. Our experiments show that our GeePS-supported Caffe scales almost linearly from the original single-machine Caffe (13$\times$ more training throughput with 16 machines), and is able to efficiently support neural networks that do not fit in GPU memory. \\
    \emph{Work published at EuroSys'16.} \\
    \emph{GeePS open-sourced at \url{https://github.com/cuihenggang/geeps}}
\vspace{.1in}
\item[] {\bf MLtuner: Automatic ML tuning.}
    ML tasks often require the selection and tuning of many
    \emph{training tunables}---such as
    the learning rate, the mini-batch size, and the data
    staleness bound---that have a significant impact on the performance.
    We designed a MLtuner system that
    is able to automatically tune settings for those training tunables.
    It can be linked with existing ML systems, such as parameter servers.
    MLtuner uses efficient snapshotting and optimization-guided
    online trial-and-error to find good initial tunable settings as well
    as to re-tune settings during execution.
    Our experiments with
    four real ML tasks, including deep learning and matrix factorization,
    show that MLtuner automatically enables
    performance within 40--178\% of having oracle knowledge of
    the best tunable settings, and outperforms oracle when no single set of
    settings are best for the entire execution. \\
    \emph{Work under submission.} \\
\end{innerlist}

\halfblankline

\textbf{Research Intern} \hfill {May 2014 to Aug 2014}\\
System Group, HP Labs\\
Mentor: Kimberly Keeton
\begin{innerlist}
\item[] {\bf Aperture: Ingest-Time Transformation for Big Time Series Data.}
    Aperture is a database system for big time series data analytics,
    and is able to support interactive analytical queries
    with sub-second latencies on both recent data and historical data.
    Our approach is to apply \emph{transformations} to the data at ingest time,
    so that future queries can be answered using the transformed data with much lower latencies.
    We have implemented several representative use cases in our system.
    For example, wavelet transformation can be used to
    compactly represent numerical time-series data,
    and analytical queries, such as correlation search, can be directly answered
    from the wavelet coefficients, thus with much lower latency.
    Count-min sketching can be used to represent massive counting information,
    such as occurrence counts of IP addresses.
    Ingest-time transformation can also be used to do anomaly detection.
    Our experiments show that our approach can speed up many analytical query benchmarks
    by over an order of magnitude, with minimal impact on ingest throughput. \\
    \emph{Work published at SoCC'15}.
\end{innerlist}

\section{Conference Talks}

GeePS: Scalable Deep Learning on Distributed GPUs with a GPU-Specialized Parameter Server \\
    In \emph{EuroSys 2016}, London, UK \hfill Apr 2016
\vspace{.1in}

Using Data Transformations for Low-latency Time Series Analysis \\
    In \emph{SoCC 2015}, Kohala Coast, HI \hfill Aug 2015
\vspace{.1in}

Exploiting Iterative-ness for Parallel ML Computations \\
    In \emph{SoCC 2014}, Seattle, WA \hfill Nov 2014
\vspace{.1in}

% Exploiting Iterative-ness for Parallel ML Computations \\
    % In \emph{Parallel Data Lab Retreat}, Bedford Springs, PA \hfill Oct 2014
% \vspace{.1in}

% Ingest-time Transformations in Time Series Database \\
    % In \emph{Parallel Data Lab Retreat}, Bedford Springs, PA \hfill Oct 2014
% \vspace{.1in}

Exploiting Bounded Staleness to Speed Up Big Data Analytics \\
    In \emph{ATC 2014}, Philadelphia, PA \hfill Jun 2014
\vspace{.1in}

LazyTable: Distributed Machine Learning with the Stale Synchronous Parallel Model \\
    In \emph{SOSP 2013 WIP Talk}, Farmington, PA \hfill Nov 2013
\vspace{.1in}

% Exploiting Bounded Staleness to Speed up Big Data Analytics \\
    % In \emph{Parallel Data Lab Retreat}, Bedford Springs, PA \hfill Oct 2013

\pagebreak
\section{Publications}

\vspace{-.1275in}
\begin{bibsection}
    \item Aaron Harlap, {\bf Henggang Cui}, Wei Dai, Jinliang Wei,
        Gregory R. Ganger, Phillip B. Gibbons, Garth A. Gibson, and Eric P. Xing.\\
        Addressing the Straggler Problem for Iterative Convergent Parallel ML.\\
        In \emph{ACM Symposium on Cloud Computing (SoCC'16)},
        2016.
    \item {\bf Henggang Cui}, Hao Zhang, Gregory R. Ganger, Phillip B. Gibbons, and Eric P. Xing.\\
        GeePS: Scalable Deep Learning on Distributed GPUs with a GPU-Specialized Parameter Server.\\
        In \emph{ACM European Conference on Computer Systems (EuroSys'16)},
        2016.
    \item {\bf Henggang Cui}, Kimberly Keeton, Indrajit Roy, Krishnamurthy Viswanathan, and Gregory R. Ganger.\\
        Using Data Transformations for Low-latency Time Series Analysis.\\
        In \emph{ACM Symposium on Cloud Computing (SoCC'15)},
        2015.
    \item Jinliang Wei, Wei Dai, Aurick Qiao, Qirong Ho, {\bf Henggang Cui},
        Gregory R. Ganger, Phillip B. Gibbons, Garth A. Gibson, Eric P. Xing.\\
        Managed Communication and Consistency for Fast Data-Parallel Iterative Analytics.\\
        In \emph{ACM Symposium on Cloud Computing (SoCC'15)},
        2015.
    \item {\bf Henggang Cui}, Alexey Tumanov, Jinliang Wei, Lianghong Xu, Wei Dai,
        Jesse Haber-Kucharsky, Qirong Ho, Gregory R. Ganger, Phillip B. Gibbons, Garth A. Gibson, and Eric P. Xing.\\
        Exploiting Iterative-ness for Parallel ML Computations.\\
        In \emph{ACM Symposium on Cloud Computing (SoCC'14)},
        2014.
    \item {\bf Henggang Cui}, James Cipar, Qirong Ho, Jin Kyu Kim, Seunghak Lee, Abhimanu Kumar,
        Jinliang Wei, Wei Dai, Gregory R. Ganger, Phillip B. Gibbons, Garth A. Gibson, and Eric P. Xing.\\
        Exploiting Bounded Staleness to Speed Up Big Data Analytics.\\
        In \emph{USENIX Annual Technical Conference (ATC'14)},
        2014.
    \item Qirong Ho, James Cipar, {\bf Henggang Cui}, Jin Kyu Kim, Seunghak Lee,
        Phillip B. Gibbons, Garth A. Gibson, Gregory R. Ganger, and Eric P. Xing.\\
        More Effective Distributed ML via a Stale Synchronous Parallel Parameter Server.\\
        In \emph{Neural Information Processing Systems (NIPS'13)},
        2013.
    \item {\bf Henggang Cui}, Danielle Rasooly, Moises R. N. Ribeiro, and Leonid Kazovsky.\\
        Optically Cross-Braced Hypercube: A Reconfigurable Physical Layer for
        Interconnects and Server-Centric Datacenters.\\
        In \emph{Optical Fiber Communication Conference and Exposition (OFC/NFOEC)},
        2012.
    \item Dan Li, {\bf Henggang Cui}, Yan Hu, Yong Xia, and Xin Wang.\\
        Scalable Data Center Multicast using Multi-class Bloom Filter.\\
        In \emph{19th IEEE International Conference on Network Protocols (ICNP'11)},
        2011.
\end{bibsection}

\section{Awards}

\textbf{Student Awards --- Tsinghua University}
\begin{innerlist}
\item[] Outstanding Graduate Award of Tsinghua University  \hfill Jul 2012
\item[] Outstanding Graduate Award of Beijing, China \hfill Jul 2012
\item[] Outstanding Diploma Thesis Award of Tsinghua University \hfill Jul 2012
\item[] National Scholarship of China \hfill Nov 2011
\item[] National Scholarship of China \hfill Nov 2010
\end{innerlist}

\halfblankline

\textbf{Travel Awards}
\begin{innerlist}
\item[] EuroSys 2016, London, UK \hfill Apr 2016
\item[] SoCC 2014, Seattle, WA \hfill Nov 2014
\item[] ATC 2014, Philadelphia, PA \hfill Jun 2014
\item[] SOSP 2013, Farmington, PA \hfill Nov 2013
\end{innerlist}

\section{Courses}

\textbf{Carnegie Mellon University (GPA 3.93/4)}

18749 Building Reliable Distributed Systems, Fall 2015. Course project:
    \begin{innerlist}
    \item[] Elastic machine learning on temporarily available/affordable resources
    \end{innerlist}
\vspace{.1in}

15712 Advanced Topics in Operating Systems, Fall 2014. Course project:
    \begin{innerlist}
    \item[] Straggler mitigation in parallel machine learning
    \end{innerlist}
\vspace{.1in}

10780 Graduate Artificial Intelligence, Spring 2014. Course project:
    \begin{innerlist}
    \item[] An online method for adaptive data staleness tuning
    \end{innerlist}
\vspace{.1in}

15799b Advanced Topics in Database Systems, Fall 2013. Course project:
    \begin{innerlist}
    \item[] Iterativeness-aware optimization for big data analytics
    \end{innerlist}
\vspace{.1in}

18746 Storage Systems, Spring 2013. Course project:
    \begin{innerlist}
    \item[] Cloudfs, a hybrid file system integrating solid-state devices and cloud storage
    \end{innerlist}
\vspace{.1in}

15750 Graduate Algorithms, Spring 2013.
    \begin{innerlist}
    \item[]
    \end{innerlist}

15799a Advanced Topics in OS Engineering, Fall 2012. Course project:
    \begin{innerlist}
    \item[] 1. Pebbles, implementing a Unix-like kernel from scratch
    \item[] 2. PebPeb, paravirtualization of Pebbles kernel
    \end{innerlist}
\vspace{.1in}

10701 Machine Learning, Fall 2012. Course project:
    \begin{innerlist}
    \item[] Implementation and comparison of parallel EM on GraphLab, Spark, and Piccolo
    \end{innerlist}
\vspace{.1in}


\section{Teaching Experience}

\textbf{Teaching Assistant}

18746/15746 Storage Systems  \hfill Spring 2015 and Fall 2016
\vspace{.1in}

% 18746/15746 Storage Systems  \hfill Fall 2016
% \vspace{.1in}


\section{Skills}

\textbf{Computer Programming}
\begin{innerlist}
    \item[] C$+$$+$, Python, C, CUDA, Java, Matlab, Shell, Perl, Scala
\end{innerlist}
\vspace{.1in}

\textbf{Software Systems}
\begin{innerlist}
    \item[] Git, Linux, LaTeX, SVN
\end{innerlist}
\vspace{.1in}

\textbf{Big data Systems}
\begin{innerlist}
    \item[] Caffe, GraphLab, MPI, Spark, HBase, OpenTSDB, Hadoop
\end{innerlist}
\vspace{.1in}

\end{document}

%%%%%%%%%%%%%%%%%%%%%%%%%% End CV Document %%%%%%%%%%%%%%%%%%%%%%%%%%%%%

%----------------------------------------------------------------------%
% The following is copyright and licensing information for
% redistribution of this LaTeX source code; it also includes a liability
% statement. If this source code is not being redistributed to others,
% it may be omitted. It has no effect on the function of the above code.
%----------------------------------------------------------------------%
% Copyright (c) 2007, 2008, 2009, 2010, 2011 by Theodore P. Pavlic
%
% Unless otherwise expressly stated, this work is licensed under the
% Creative Commons Attribution-Noncommercial 3.0 United States License. To
% view a copy of this license, visit
% http://creativecommons.org/licenses/by-nc/3.0/us/ or send a letter to
% Creative Commons, 171 Second Street, Suite 300, San Francisco,
% California, 94105, USA.
%
% THE SOFTWARE IS PROVIDED "AS IS", WITHOUT WARRANTY OF ANY KIND, EXPRESS
% OR IMPLIED, INCLUDING BUT NOT LIMITED TO THE WARRANTIES OF
% MERCHANTABILITY, FITNESS FOR A PARTICULAR PURPOSE AND NONINFRINGEMENT.
% IN NO EVENT SHALL THE AUTHORS OR COPYRIGHT HOLDERS BE LIABLE FOR ANY
% CLAIM, DAMAGES OR OTHER LIABILITY, WHETHER IN AN ACTION OF CONTRACT,
% TORT OR OTHERWISE, ARISING FROM, OUT OF OR IN CONNECTION WITH THE
% SOFTWARE OR THE USE OR OTHER DEALINGS IN THE SOFTWARE.
%----------------------------------------------------------------------%
